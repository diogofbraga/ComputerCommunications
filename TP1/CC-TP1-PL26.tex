\documentclass{llncs}
\usepackage{times}
\usepackage[T1]{fontenc}

% Comentar para not MAC Users
%\usepackage[applemac]{inputenc}

\usepackage{a4}
%\usepackage[margin=3cm,nohead]{geometry}
\usepackage{epstopdf}
\usepackage{indentfirst}
\usepackage{graphicx}
\usepackage{float}
\usepackage{fancyvrb}
\usepackage{amsmath}
\usepackage{array}
%\renewcommand{\baselinestretch}{1.5}


\begin{document}
\mainmatter
\title{TP1 - Protocolos da Camada de Transporte}

\titlerunning{TP1 - Protocolos da Camada de Transporte}

\author{Diogo Braga \and João Silva}

\authorrunning{Diogo Braga \and João Silva}

\institute{
University of Minho, Department of  Informatics, 4710-057 Braga, Portugal\\
e-mail: \{a82547,a82005\}@alunos.uminho.pt\\
PL2, Grupo 6
}

\date{}
\bibliographystyle{splncs}

\maketitle

\section{Questão 1}

\begin{center}
\begin{tabular}{ | m{3cm} | m{3cm} | m{3cm} | m{3cm} | m{3cm} |}
\hline
 \textbf{Comando usado (aplicação)} & \textbf{Protocolo de Aplicação(se aplicável)} & \textbf{Protocolo de transporte(se aplicável)} & \textbf{Porta de atendimento(se aplicável)} & \textbf{Overhead de transporte em bytes(se aplicável)} \\
 \hline
 \textbf{Ping} & -------------------------- & -------------------------- & -------------------------- & -------------------------- \\
 \hline
 \textbf{traceroute} & -------------------------- & -------------------------- & -------------------------- & -------------------------- \\
 \hline
 \textbf{telnet} & Telnet & TCP & 23 & 20 \\
 \hline
 \textbf{ftp} & FTP & TCP & 21 & 32 \\
 \hline
 \textbf{Tftp} & TFTP & UDP & 69 & 22 \\
 \hline
 \textbf{browser/http} & HTTP & TCP & 80 & 32 \\
 \hline
 \textbf{nslookup} & DNS & UDP & 53 & 47 \\
 \hline
 \textbf{ssh} & SSH & TCP & 22 & 32 \\
 \hline
\end{tabular}
\end{center}

\subsection{Ping}

\subsection{Traceroute}

\subsection{Telnet}

\subsection{FTP}

\subsection{TFTP}

\subsection{HTTP}

\subsection{Nslookup}

\subsection{SSH}


\section{Questão 2}

\section{Questão 3}
\subsection{(i) uso da camada de transporte}
Perante os resultados obtidas na Questão 1, podemos concluir que, tanto o FTP, como o HTTP e o SSH utilizam o TCP como camada de transporte, enquanto o TFTP utiliza o UDP para essa função.

Isto leva-nos a concluir que no TFTP existe um menor controlo do envio dos dados por parte da camada de transporte, visto este ser antes controlado pela camada de aplicação.

No caso das restantes aplicações em causa, como é usado o TCP sabemos que o transporte, sendo orientado à conexão, é realizado de forma fiável, pois efetua também controlo de erros, de fluxo e de congestão.

\subsection{(ii) eficiência na transferência}
No protocolo de aplicação TFTP (uso do UDP),

\subsection{(iii) complexidade}
Perante o observado, por exemplo na Questão 2, concluimos que a complexidade de usar o UDP é consideravelmente inferior à de usar o TCP. Nesse caso, o FTP necessita de estabelecer conexão, transferir os dados, e realizar a desconexão. Ainda na mesma questão, conseguimos concluir que o TFTP, através do UDP, apenas realiza a transferência dos dados.

...

\subsection{(iv) segurança}


\section{Questão 4}

\section{Conclusões}

\end{document}
